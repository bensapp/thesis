\chapter{Related work}

In this section we survey a variety of methods for 2D human pose estimation in 
both single frames and video.  The majority of this work is on {\em part-based} 
models, which decompose and reason about basic units of pose.  A few 
non-part-based approaches are mentioned in \secref{nonpart}.  We omit work 
using other more powerful sensors such as motion capture environments or 
multiple cameras, since many of the significant issues in 2D human pose 
estimation are not present in these domains (\eg, clutter, viewpoint and 
occlusion issues).

\section{Single-frame parts-based models}\label{sec:partsbased}

The idea that real-world objects captured in 2D images can be modeled as 
collections of interrelated atomic parts goes back to the foundational work of 
\citet{binford71}. The term ``pictorial structures'' was introduced by 
\citet{fischler1973ps}.  In this work, the authors lay out the basic PS model 
described in \secref{ps}, an aggregation of individual local part scores and 
pairwise part-part scores which depend only on the displacement between 
connected parts.  The standard dynamic programming solution is presented here 
(see \secref{inference}), quadratic in the number of parts.  

The pictorial structures model was made popular in recent literature by 
\citet{felz05}, who contributed a linear time message passing algorithm using 
distance transforms (explained in \secref{dt}) which made tree-structured 
models tractable.  This led to an explosion of work using and refining this 
basic PS model:

\citet{devaparse} introduced discriminative training of edge template filters 
to improve upon previously hand-crafted templates, and proposed an iterative 
estimation procedure which uses the current guess at the foreground color 
distributions in subsequent parses.  \citet{ferrari08} provide an end-to-end 
multi-step pipeline to parse upper body pose in video and whittle down the 
number of possibilities: first detect people, estimate their color, reject 
background hypotheses using a conservative color-based graphcut, then track 
remaining candidates through time.  \citet{eichner-tr} builds on 
\citet{ferrari08} and \citet{devaparse} to make a competitive state-of-the-art 
system.  \citet{andriluka09} improve upon the basic PS model by learning more 
powerful part filters using Adaboost classifers on top of shape context to 
represent each part.

All of these models rely heavily on the facts that (1) the part interactions 
form a tree and (2) these interactions must be simple functions of geometry 
only.  If these are invalidated, obtaining the best solution from the model is 
intractable (\secref{inference}).   From a certain viewpoint, the rest of the 
literature discussed here is an innovation in response to at least one of these 
limitations, which lead to the following shortcomings:

\begin{itemize}
\item[S-A:] {\em Lack of important part-pair relationships.} For example, to 
mantain acyclicity, typically the left and right arms are left unconnected.  
However, this omits useful cues for parsing---\eg, to express the fact that 
arms are likely the same color, or do not often lie on top of each other.  

\item[S-B:] {\em Lack of larger and richer relationships.}  Modelers would also 
like to design cost functions that consider more than two parts at a time (and 
informed by more than just geometry).  For example, one would like to embed in 
the cost function answers to questions such as "is this a realistic half-body 
with respect to countours in the image?"  Representing larger collections of 
parts \naively leads to a combinatorial explosion of possibilities.

\item[S-C:] {\em A ``one-size-fits-all' representation of appearance and pose.} 
In a single tree with parameters describing part appearance and part geometry, 
it is difficult to capture the wide variability in human pose found in nature.  
For example, a single tree model must have a very general template for the 
lower arm to capture all appearance changes due to clothing, lighting, body 
size and articulation. It must also represent the wide range of lower arm-upper 
arm geometries with a single (typically unimodal) angular distribution.


\end{itemize}

\subsection{Dealing with cyclic models} In a desire to overcome shortcoming 
S-A, some researchers consider cyclic, single frame graphs.  In a cyclic, or 
loopy, graph, inference is $\#P$-complete~\citep{koller-book}, without a known 
polynomial-time algorithm.  \citet{ddtran} consider a fully connected graph of 
limb interactions, and use local greedy search as an approximate inference 
technique.  Quite recently \citet{min-bb} use a fully-connected graph using the 
output of our cascade system (\secref{CPS}), and improve  upon our results 
which use a tree model.  They use branch-and-bound techniques and an order of 
magnitude more time for inference to obtain exact solutions (roughly 20 minutes 
versus 10 seconds).

Many other approximate inference techniques have been applied to cyclic graphs 
in video, see \secref{rel-video}.

\subsection{A family of trees}\label{sec:tree-fam}
Describing pose with multiple trees can solve issues with shortcomings S-A and 
S-C.   Using multiple trees, we can cover all part relationships desired in at 
least some tree, and maintain acyclicity in all trees.  Or, multiple trees can 
share the same graph structure, but each tree (or subcollection) can model a 
subspace of the pose space (\eg, an arms-crossed model; a profile model). Tree 
beliefs are typically accumulated or maxed over to determine a final pose.  In 
either case, such models are sometimes called ``mixture-of-trees'' models.
  
\citet{wang2008multiple} use multiple tree models to capture previously unused 
part relationships to reason about occlusion and ``double counting'' (left and 
right symmetric parts both attaching to the most likely part image evidence).  
Our ESM model~\secref{stretchable} uses these types of cues and more.  
\citet{ioffe2001} also use multiple trees to reason about occlusion in 
video---one tree for each common occlusion scenario.  \citet{everingham2011} 
and \citet{ramanan-faces} use dozens of tree models for full body and face 
parsing, respectively, to capture broad pose categories such as ``upside down'' 
body or ``left 3/4 profile'' face.  In practice, mode definitions are data 
driven and not necessarily semantically interpretable.

\subsection{Multimodal compositional tree models}
To address shortcomings S-B, many works have begun to look at additional 
``parts'' which are actually compositions of atomic parts (\eg lower arm plus 
upper arm induce a full arm part).  The compositional parts connect to simpler 
parts and model geometric consistency constraints 
\citep{wang2011,sun2011,deva2011}.  By virtue of necessity, the appearance 
terms of compositional parts must be multimodal----it would be impossible to 
capture all appearances of an articulated half-body with a single part 
template.  The typical approach is to pre-define a handful of discrete modes 
based on training data and thus have a collection of mode possibilities for 
each part. During inference, the correct mode is inferred as a state in the 
model or maxed out.  This type of multimodal modeling extends and has shown to 
be very valuable even for only atomic parts \citep{deva2011}.  In fact, even a 
basic PS model which searches over rotations for each part can be thought of as 
an atomic part multimodal model, where each part orientation is a model, and 
all modes share the same appearance parameters (the rotated template).

One can think of a family of trees that captures different portions of pose 
space per tree (\secref{tree-fam}) in this way: it is a multimodal model at the 
most global level.  We take this view in \secref{llps-rel} and give a more 
in-depth discussion of recent work in this area there, including our LLPS 
model.


\section{Bottom-up driven approaches}
Decomposing the body into parts and only local interactions is one way of 
dealing with the combinatorial number of possibilities of human layout. An 
orthogonal approach is to keep the number of full pose possibilities manageable 
by growing them incrementally from bottom-up information, and rejecting 
unlikely pose proposals along the way.  This is attractive over parts-based 
models in that, in intermediate stages, larger context can be leveraged to 
evaluate the quality of a larger collection of parts. This is a remedy to 
shortcomings S-A and S-B described in \secref{partsbased}.

One of the earliest approaches in this vein was the idea of ``body plans'' 
\citep{bodyplans}, models of object layout defining what parts of an object can 
be grouped together, at what stage of grouping, and how.  This idea was 
revisited in \citep{mori04}, which supports a heuristic set of rules to group 
superpixels into limbs, then combine limbs into partial and finally full 
configurations.  \citet{praveen07} furthered this line of work by evaluating 
intermediate proposals by shape matching against a database of exemplar shapes 
for each stage.

The bottom-up approach holds potential for more powerful representations of 
large collections of parts than local parts-based models.  On the other hand, 
the inference procedure in all these works is greedy and hand-tuned or learned 
at each grouping stage separately.  It's not clear what exact cost function is 
being minimized, or whether they reach a global or local solution.


\section{Temporal Models}\label{sec:rel-video}

Estimating pose in a sequence of images allows us to exploit inter-frame cues 
such as smoothness of motion and appearance.  This has the potential to improve 
upon pose estimation independently in each frame.  However, it introduces an
additional challenge: tracking multiple parts over time leads to 
intractability.  In the graphical model literature, this is known as {\em 
entanglement } \citep{koller-book}, a property which implies that inference 
becomes exponential in the number of parts times the length of the image 
sequence.  This is a major computational hurdle to overcome.  The works listed 
here provided different approximate solutions to this exponential roadblock, in 
order to exploit temporal continuity cues for better pose estimation.

\subsection{Approximate inference in cyclic networks}
Some rely on standard approximate inference techniques, which often result in 
expensive inference and poorly understood behaviors.  \citet{ferrari08} use 
loopy belief propagation to incorporate geometric continuity through time.  
Loopy belief propagation involves iteratively propagating beliefs around the 
metric until convergence is reached.  Convergence, however, is not guaranteed, 
and the number of belief propagation steps required in practice is orders of 
magnitude more than a tree model.  

Loopy belief propagation is a {\em variational inference} method which can be 
interpreted as attempting to maximize a simpler distribution than the true 
Markov Random Field distribution of the model~\citep{jaakkola2000}.  Another 
variational approach proposed by \citet{ioffe2001} is direct coordinate ascent 
of the distribution.  In their case, their coordinate ascent is guided by the 
problem: they alternate between maximizing assignments within frames, holding 
inter-frame beliefs fixed, then maximizing between frames, holding intra-frame 
beliefs fixed.  This can be viewed as a application-specific constrained loopy 
belief propagation to achieve a local maxima.

\citet{sigal2004tracking} use non-parametric belief propagation with learned 
motion distributions for temporal edges. In \citet{sigal2004measure}, a 
continuous state space model is employed.  Continuous-state models require 
local scores with analytical normalization (such as Gaussian distributions), or 
approximate inference frameworks, such as non-parametric belief propagation (a 
form of sampling) used by \citet{sigal2004measure}.

Particle filtering is another form of approximate inference.  Most particle 
filtering methods only support {\em forward}, or {\em online} inference in 
which the whole video sequence cannot be taken into account (and thus future 
information cannot be used to help in earlier video frames).  The Condensation 
algorithm by \citet{isard98} tracks multiple contours (parametrized by 
B-splines) via particle filtering.  \citet{nevatia2011} consider a cyclic 
intra-frame model in conjunction with pose and action-part modes propagated 
through time, also approximately tracked with particles, using the same 
algorithm as \citet{sigal2004measure}.

In \secref{stretchable}, we study another form of approximate inference, dual 
decomposition \citep{komodakis2007}, a principled approximation framework with 
convergence guarantees.  Dual decomposition decomposes inference in a cyclic 
network to inference in a set of tree ``slave'' networks, and iterates to make 
the slaves agree on a solution.  We also provide a similar but much simpler 
approximation method that is computationally cheaper with no reduction in 
performance in \secref{stretchable-inference}.


\section{Tracking-by-detection}  A second approach to approximate 
inference---where we consider all frames or all frames preceeding the current 
frame---is instead to make hard decisions in every frame as to a limited set of 
hypotheses.  This keeps the number of pose hypotheses from growing 
exponentially with time, and allows us to use much more powerful features 
between a pair of frames.  However, it is impossible to recover from mistakes 
made in early frames.

\citet{ren07} is a good example of this paradigm, using powerful reasoning 
(graph matching with color coherence) between frames to establish 
correspondences between the assumed correct pose in the current frame and 
segments in the next frame.  In this system, there is no top-down model the 
tracked object, so drifting away from the correct pose eventually is 
inevitable.  In earlier work in the same spirit, \citet{bregler98} propagate an 
assumed correct pose in the current frame to the next by doing local gradient 
descent to fit an exponential map representation of a kinematic chain of parts.

In the ``strike a pose'' work by \citet{strikeapose}, a rigid anchor pose is 
first detected with high confidence.  From it, more discriminative detectors 
based on color are learned from it to better detect poses in neighboring 
frames.  The detection in each frame is done independently, but informed by the 
hard decision made by already detected poses. \citet{andriluka2008people} 
propose a ``people-tracking-by-detection'' method which first detects people 
independently in every frame, then reasons about people tracks and their latent 
walking modes through a hierarchical Gaussian process latent variable model.



\section{Holistic approaches}\label{sec:nonpart}

Despite limitations, parts-based models allow for decomposing the problem from 
exponentially many possibilities to a quadratic number for each pair of parts.  
Atomic parts can employ appearance models with some degree of genericness, 
enabling the models to generalize to unseen poses in the future.

An entirely different approach is to tackle the problem holistically without 
decomposing the problem into smaller, conditionally independent pieces.  These 
methods attempt to map directly from image to part locations, typically through 
the use of exemplars. Possibly the simplest is the work of \citet{mori02} which 
finds a nearest neighbor based on shape context, and transfers joint locations.  
\citet{toyama2002} consider modeling pose as a mixture of exemplar similarities 
using chamfer distance to define a metric space.  \citet{shak2003} treat 
exemplar distance as a locality sensing hashing problem.  More recently, 
\citet{taylorpose} learn a non-linear embedding so that nearest-neighbor pose 
classification is effective at predicting hand locations, using a nearest 
neighbor database of $40,000$ images.

Finally, \citet{ionescu2011} learns a non-linear regression from figure-ground 
hypotheses directly to a vector of 3D joint locations.  The regression of 
different joints is non-linearly decorrelated by embedding in a latent space 
via Kernel Dependency Estimation, with a kernel based on similarity between 
estimated figure-ground masks and groundtruth masks generated from a synthetic 
3D model rendered in a variety of poses. 

All of the works mentioned here have not been applied to recent ``in the wild'' 
datasets that we consider in this thesis.  It remains an open question whether 
such exemplar-based methods can capture all the variations present in such 
challenging datasets.  Parts-based models can achieve a much higher degree of 
generality due to their decomposable nature.


