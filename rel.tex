\chapter{Related work}

In this section we survey a variety of methods for 2D human pose estimation.  
We focus here on methods that are meant to run in a general setting (\ie in the 
wild).  In particular, we don't address models that make additional assumptions 
about the environment, such as known background, known person or clothing, or 
motion capture systems.


% classic PS stuff
\mypar{Tree models} The classical PS model described in \secref{ps} was first 
introduced by \citet{fischler1973ps} and serves as a canonical example of a 
parts-based model with pairwise interactions between parts.  Importantly, the 
graph of interactions forms a tree, which allows for efficient inference in 
time quadratic in the number of states.  A method for max-inference that is 
linear in the number of states was introduced by \citet{felz05}, and since then 
there has been an explosion of work focusing on improving this model by 
training it discriminatively~\citep{devacrf,andriluka09} and improving upon 
individual part localization capabilities via edge 
descriptors~\citep{andriluka09} and color \citep{eichner09}.

%more interactions
\mypar{Beyond tree structure}
A few past works have looked at going beyond a tree model of pose in a single 
image.  In a cyclic, or loopy, graph, inference is 
$\#P$-complete~\citep{koller-book}, without a known polynomial-time algorithm.  
\citet{ddtran} consider a fully connected graph of limb interactions, and use 
local greedy search as an approximate inference technique.  
\citet{wang2008multiple} consider multiple trees over parts, where different 
edges can express different geometric and occlusion constraints, and the union 
of all edges forms a cyclic graph.  They use a learned weighted sum of 
individual tree beliefs at test time.  Quite recently \citet{min-bb} use a 
fully-connected graph using the output of our cascade system (\secref{CPS}), 
and improve  upon our results which use a tree model.  They use 
branch-and-bound techniques and an order of magnitude more time for inference 
to obtain exact solutions (roughly 20 minutes versus 10 seconds).

%time graphs, approximate inference
\mypar{Temporal models}
In temporal processing, pairwise models resort to approximate inference 
techniques to deal with intractable inference.  \citet{ferrari08} use loopy 
belief propagation, with temporal interactions expressing spatial continuity 
between consecutive frames.  In 3D, \citet{sigal2004tracking} use 
non-parametric belief propagation with learned motion distributions for 
temporal edges.

%hierarchical & multi-modla
\mypar{Hierarchical and multi-modal models}
Another extension to basic tree models are hierarchical models of pose, where 
the hierarchy is over part granularity (\eg left lower arm, left upper and 
lower arm, left half body, full body). \citet{wang2011,sun2011} both developed 
such models, targeted towards activity recognition.  These models discretize 
the coarser parts into a set of possible modes to better model variations of 
larger deformable parts.  Learning different modes is also done 
by~\citet{deva2011}, a non-hierarchical model over joints and limb midpoints, 
as well as~\citet{everingham2011}, who clusters global pose into modes to learn 
different tree models.

\mypar{Non-discrete models}
An alternative approach is to use a continuous state space instead of a 
discrete one.  In order to make inference tractable, such methods require 
restricted models with analytical normalization (such as a Gaussian 
distributions), or approximate inference frameworks, such as non-parametric 
belief propagation, a form of sampling~\citep{sigal-thesis}.  

\mypar{Bottom-up and greedy methods}
All of the above approaches deal with the combinatorially-many possible poses 
by aggregating information or heuristic searching in a clever manner.  A 
different approach is to instead propose a small set of likely hypotheses based 
on bottom-up perceptual grouping principles \citep{mori04,praveen07}.  
\citet{mori04} use bottom-up saliency cues, for example strength of supporting 
contours, to generate limb hypotheses. They then prune via hand-set rules based 
on part-pair geometry and color consistency.  \citet{ren07} track people 
without a model of a person, using a region segmentation of each frame, and a 
one-to-many segment assignment optimization problem between frames to make 
greedy tracking decisions.
