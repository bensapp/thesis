\begin{tabular}{|c|c|c|c|c|c|}
\hline
\multirow{3}{*}{cascade level}   & state        & \multicolumn{2}{|c|}{\# 
states in the}        &  \multirow{3}{*}{ } state space       & near-correct \\\cline{3-4}
 &       dimensions &original   & pruned        & reduction     &  lower arms  \\
 &       & space        & space & \%    & unpruned (PCP) \\

\hline
\hline
0        & 10x10x12      & 153600        & 1200 &        00.00 & 100  \\
\hline
1        & 10x10x24      & 72968 &      1140  &  52.50   & 76.6 \\
\hline
3        & 20x20x24      & 6704  & 642  &  95.64         & 72.3 \\
\hline
5        & 40x40x24      & 2682  & 671   & 98.25         & 70.5 \\
\hline 
7        & 80x80x24      & 492   & 492&          99.67   & 68.4 \\
\hline
\hline
detection pruning        & 80x80x24      & 492   & 492&          99.67   & 58.6 \\
%pruning         &       &       &&              &  \\
\hline
\end{tabular}

