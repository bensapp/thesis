\chapter{Machine learning classification}

In this section we lay out the basic definitions and tools necessary for 
data-driven modeling of {\em classification problems}.  In the most general 
setting, the goal is to learn a mapping, or classifier $h$ from a hypothesis 
class $\cH$ from a set of input variables $x$ to a set of discrete output 
variables $y$.   

We concern ourselves with a specific setting: $x$ lies in a (possibly 
high-dimensional) vector space (most commonly in this work, the input image 
pixels) and the target lies in a discrete $n$-dimensional space: 
$$y \in \{0,1,\ldots,k\}^n.$$  

We will refer to the set $\{0,1,\ldots,k\}$ as the set of labels, {\em label 
set}, or {\em state space} for a dimension of $y$.  Each dimension of $y$ will 
be referred to as $y_i$.  For example, in a simple image classification task to 
determine whether an image has a dog in it, $x$ could be an image's pixels, and 
$y \in \{dog, not~dog\} \cong \{0,1\} $, an instance of {\em binary 
classification}. In this case, $k = 2$ and $n 
= 1$.  In {\em multiclass classification}, $k>2$ and $n = 1$, \eg, predicting 
the handwritten digits $0,\dots,9$ or the weather $\{sunny, cloudy, rainy\} 
\cong \{0,1,2\}$.  Finally, in {\em structured prediction}, the output $y$ is a 
vector: $n>1$.

In this section we will describe each in turn building up, specifying the 
following necessary components for classification problems.

\begin{itemize}
\item An {\bf inference procedure} to determine the most likely label for a 
fixed test example $x$: \begin{equation}
h(x) \triangleq \argmax_{y \in \cY} h(x,y).
\end{equation}
\item The {\bf hypothesis class $\cH$} from which to obtain our classifier $h$.
\item  A {\bf learning loss function} $\cL(\cdot)$ which assesses the quality 
of any particular classifier $h \in \cH$.
\item A {\bf learning algorithm} to find the minimizer of $\cL$: 
\begin{equation}
h = \argmin_{h' \in \cH} \cL(h').
\end{equation}
\end{itemize}

\begin{table}[tb!]
\centering
\begin{tabular}{| c | c | }
\hline
symbol & definition \\
\hline
\hline
$x \in \cX = \reals^d$ & input variables\\
$y \in \cY =\{0,1,\ldots,k\}^n$ & output variables \\
$h(x,y): \cX \times \cY \mapsto \cY$ & classifier \\
\hline
\end{tabular}
\caption[Machine learning notation]{Machine learning notation.}
\label{tab:notation}
\end{table}



\section{Linear classifiers}
There are many possible parametric and non-parametric choices of hypothesis 
classes $\cH$ to consider.  One of the easiest to represent, learn and analyze 
is the hypothesis class of {\em generalized linear models}:
\begin{equation}
h(x,y) = \sum_{i=1}^d w_i f(x,y)\ = \w \cdot \f(x,y) 
\end{equation}
where $\w \in \reals^d$ is a vector of linear parameters of our model and 
$\f(x,y) \in \reals^d$ is a vector of features that depend on the input and 
output variables.  Thus the model is simply a weighted sum of features to 
obtain a real valued score, and $\cH = \reals^d$.


\section{Structured models}

The inference problem is to determine the highest scoring possibility out of 
all possible outputs in $\cY$: \begin{equation} h(x) = \argmax_{y \in \cY} 
h(x,y).  \end{equation}  When $\cY$ is low-dimensional, brute-force search is 
easy enough: evaluate $h(x,y)$ for all possible $y$, and return the highest 
scoring.

For binary classification, this requires only a single dot product in $d$-dimensions
\begin{align}
h(x) &= \Ind\left( h(x,1) > h(x,0)\right) \\
&= \Ind\left( \w \cdot \f(x,1) > \w \cdot \f(x,0) \right) \\
&= sign\left( \w \cdot \left(f(x,1) - f(x,0)\right) \right) \\ 
&= sign\left( \w \cdot \tilde{f}(x) \right)  
\end{align}
\todo{view of this as a hyperplane...and it's good to keep this intuition even for more complex settings than binary}

In a multiclass problem with $k$ possible labels for $y$, we require in general $k$ $d$-dimensional dot products.

However, in the most general setting, $\cY$ is exponential in 
$k$: $|\cY| = k^n$, and we can't hope to enumerate all possible outputs in 
$\cY$ to find the highest-scoring.  Without any further modeling assumptions, 
there is not much we can do here.  In order to handle problems of these times, 
we impose assumptions about the {\em structure} of the output space.  

For example, imagine we are tracking the location of a robot on a map with 500 
locations over 100 time steps.  We have noisy sensor readings $x$ for every 
time step.  The output $y$ is a sequence of locations the robot visited in all 
100 time steps, one out of $\cY = 100^{500} = 10^{1000}$ possibilities.  By 
comparison, the estimated number of atoms in the universe is only $10^{80}$.  

The key insight to be made is the following: if we know where the robot is at 
time $t$, this is tremendously useful for determining the robots next location 
at time $t+1$ .  In addition, knowing where the robot was at $t-1$ also helps 
localize it at $t+1$ since we can now estimate velocity given two locations, 
but not as much.  Going further into the past, there is probably very little 
information to help localize where the robot is at time $t+1$ if we are told 
where it was many steps ago, say, $t-10$.  In general, we can make the 
simplifying assumption that {\em given the recent past, the future is 
independent of the distant past}.

In other structured problems, it also makes sense to make {\em independence 
assumptions} about which dimensions of $y$ are assumed dependent on each other and their interactions should be directly modeled. 
In handwriting recognition, adjacent letters in a document 
depend very little on letters far away in the document which correspond to different words.
In human pose estimation, knowing the location of the left shoulder is only a weak indicator of where the right wrist should be. In scene labeling, we can assume the label of the upper right image region does not depend on far away regions in the image~\citep{cour05}.

You may object that some of the assumptions made in the above examples are 
somewhat extreme.  However, they are typically seen as forgivable thanks to the 
great decrease in computational complexity they provide in the inference 
problem.  For any problem in which dimensions of $y$ can be grouped into 
overlapping dependent sets of dimensions $\{c\}$, then the linear predictor $\w 
\cdot \f(x,y)$ can be {\em decomposed} into a sum over overlapping factors.
\begin{equation}
\w \cdot \f(x,y) = \sum_{c \in \cC} \w_c \cdot \f_c(x,y_c) = \sum_{c \in \cC} \phi_c
\end{equation}
where we use the shorthand for cliques $\phi_c \defn \w_c \cdot \f(x,y_c)$.

As a simple example of this, consider our robot problem where we make the 
common assumption that the position of the robot at time $t+1$ only depends on 
the immediate past $t$.  Then the linear structured model for this problem can 
be written
\begin{equation*}
\w \cdot \f(x,y) = \sum_{t=1}^{99} \w_{t,t+1} \cdot \f(x,y_{t,t+1}) = \sum_{t=1}^{99} \phi_{t,t+1} 
\end{equation*}
Observe now how computing the maximum score of the classifier also decomposes for this simple problem:
\begin{align}
&\max_{y \in \cY} \w \cdot \f(x,y)\\ &= \max_{y_1,\ldots,y_{100}} \sum_{t=1}^{99} \phi_{t,t+1} \\
%&= \max_{y_{100}} \left( \phi_{99,100} + \max_{y_{99}} \left( \phi_{98,99} + \ldots \max_{y_3} \left(\phi_{3,4} + \max_{y_2} \left( \phi_{2,3} + \max_{y_1} \phi_{1,2} \right) \right) \ldots \right) \right) \\
&= \max_{y_{100}} \left[ \phi_{99,100} + \max_{y_{99}} \left[ \phi_{98,99} + \ldots \max_{y_3} \left[\phi_{3,4} + \max_{y_2} \left[ \phi_{2,3} + \max_{y_1} \phi_{1,2} \right] \right] \ldots \right] \right] 
%&= \max_{y_{100}} \phi_{99,100} + \max_{y_{99}} \phi_{98,99} + \ldots \max_{y_3} \phi_{3,4} + \max_{y_2} \phi_{2,3} + \max_{y_1} \phi_{1,2} 
\label{eq:nesting} \end{align} 
In~\equref{nesting}, we see that the max over each dimension of $y$ can be
nested and needs only reason over individual factors at a time.  In our robot problem,
each factor is only $500^2 = 2500$ numbers to consider and there are 99 factors. The total number of work done is $O(n \cdot k^2) \approx 250,000$ in
this case, an astronomical improvement over the complete space of $10^{1000}$.

This nice property can be generalized to any linear model that decomposes into 
factors.  We can encode the sets of factors and their overlap relations in 
general using a {\em factor graph}~\citep{koller-book}.  The number of 
different settings and representations for structured models is vast and varied 
and outside the scope of this work. 

\subsection{Pairwise structured prediction}\label{sec:pairwise-model} For our purposes, we focus on structured problems with at most {\em pairwise structure}, where $|c| \leq 2\;\; \forall c \in \cC$.  This simplification does not result in loss of generality, as factors involving 3 or more variables can always be converted into pairwise or unary factors with an expanded label set consisting of the Cartesian product of each variable's state space. For example, a factor involving the variables $y_1,y_2$ and $y_3$ with state space $\{1,\ldots,k\}^3$ can be converted into a single variable $y_{123} \in \{1,\ldots,k^3\}$.

We can represent a pairwise structured model as a graph $G = (\cV_G,\cE_G)$ where each variable $y_i$ corresponds to a vertex in the graph's vertex set $\cV_G$, and each edge in $\cE_G$ corresponds to two variables $y_i,y_j$ being involved in a pairwise factor $\phi_{ij}$.  We can then decompose our classifier into the special form
\begin{align}
\w \cdot \f(x,y) &= \sum_{i \in \cV_G} \w_i \cdot \f(x,y_i) + \sum_{i,j \in \cE_G} \w_{ij} \cdot \f(x,y_{ij}) \\
& = \sum_{i \in \cV_G} \phi_i + \sum_{i,j \in \cE_G} \phi_{ij}.
\end{align} 
There is a huge amount of literature dedicated to this form of model, originally stemming from the physics literature.  When given a probabilistic interpretation as we will see in~\secref{probinterp}, this type of model is known as a {\em pairwise Markov Random Field} (MRF).  Coming from this literature which uses this form of model to model \todo{WTF am I saying}, we refer to the different terms as {\em potentials}.  Terms of the form $\phi_i = \w_i \cdot \f_i$ we will refer to as {\em unary potentials}; $\phi_{ij} = \w_{ij} \cdot \f_{ij}$ are {\em pairwise potentials}.

When blah forms a tree, we can always exploit the structure for exact efficient inference.  Using the same trick we used in the specific example of the robot localization problem~\equref{nesting}, when the graph structure (synonomously, pairwise variable interactions) form a tree, we can perform inference by nesting the $\max$ operator over pairs of variables at a time, as well as storing the $\argmax$ maximizer of each $\max$ operation.  For completeness, the algorithm is in \algref{max-inference}, commonly referred to as max-sum {\em message passing, belief propagation}, or {\em viterbi decoding}.  The algorithm results in inference that is $O(nk^2)$ computation rather than $k^n$.

\begin{algorithm} 
\caption[max-sum inference]{Max-sum message passing to solve 
$$\argmax_{y \in \cY} h(x,y) = \argmax \sum_{i \in \cV} \phi_i + \sum_{ij \in \cE} \phi_{ij}$$} 
\label{alg:max-inference} 
\begin{algorithmic} 

\REQUIRE $ $ \\ 
Factors $\{\phi_i\}, \{\phi_{ij}\}$\\
Tree graph $G$ with (arbitrary) root node index $r$ and topological ordering $\pi$, where $\pi_n = r$.

\ENSURE $y^\star = \argmax_{y} h(x,y)$ 
\FOR{$i = \pi_1, \pi_2, \ldots, \pi_n $ }
\STATE
$m_i = \phi_i + \sum_{j \in \text{kids}(i)} m_{j \rightarrow i}$
\IF{$i == r$} \STATE \textbf{break} \ENDIF
\STATE
$p = \texttt{parent}_\pi(i) $
\STATE
$m_{i \rightarrow p} = \max_{y_i} \phi_{ip} + m_i$
\STATE
$a_i = \argmax_{y_i} \phi_{ip} + m_i$
\ENDFOR

\STATE
$y^\star_r = \argmax m_r$
\FOR{$i = \pi_{n-1},\pi_{n-2},\ldots,1$}
\STATE $y^\star_i = a_i\left[y^\star_{\texttt{parent}_\pi(i)}\right]$
\ENDFOR

\end{algorithmic} 
\end{algorithm}



\subsection{Probabilistic interpretation}\label{sec:probinterp}

\section{Inference}
\subsection{max-sum}
\subsection{max-marginals}
\subsection{probabilistic marginals}
\section{Supervised learning}
In the supervised learning setting, we have access to a training set
$S = \{(x^{(j)},y^{(j)})\}_{j=1}^m$ of training examples, assumed to be sampled 
independently, identically distributed (i.i.d.) from some true joint 
distribution $(x,y) \sim P(X,Y)$. The standard supervised learning task is to 
learn a hypothesis $h: \cX \times \cY \mapsto \cY$ that minimizes the average 0/1-error on the training set: 
\begin{align}
\cL_{01}(h,S) = \hat{\E}_{S}\left[
  \ell\left(h(X),Y\right)\right] = \frac{1}{m}\sum_{j=1}^m \Ind\left(h(x^{(j)}) = y^{(j)}\right)
\end{align}.


for some loss function $\ell: Y \times Y \rightarrow \reals^+$. The linear hypothesis class we
consider is of the form $h(x) = \argmax_y \score{y}$, where the
scoring function $\score{y} \triangleq \w \cdot \f(x,y)$ is the
inner product of a vector of parameters $\theta$ and a feature
function $\bft: X \times Y \mapsto \reals^d$ mapping $(x,y)$ pairs to
$d$ real-valued features. 


