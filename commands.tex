%% specific commands

%kronecker features

\newcommand{\Loss}{\mathcal{L}}
\newcommand{\LossZO}{\mathcal{L}_{01}}
\newcommand{\LossM}{\mathcal{L}_{\psi}}
\newcommand{\LossA}{\mathcal{L}_{\psi}}
\newcommand{\LossMAX}{\mathcal{L}^{max}_{\psi}}
\newcommand{\X}{\mathcal{X}}
\newcommand{\E}{\mathbb{E}}
\newcommand{\bw}{\mathbf{w}}
\newcommand{\bft}{\mathbf{f}}
\newcommand{\bx}{\mathbf{x}}
\newcommand{\by}{\mathbf{y}}
\newcommand{\ip}[2]{{#1}\cdot{#2}}
\newcommand{\Rgn}{\mathcal{R}}
\newcommand{\G}{\mathcal{G}}
\newcommand{\Y}{\mathcal{Y}}

\newcommand{\Ind}{\mathbf{1}}
\newcommand{\argmax}{\mathop{\arg\max}}
\newcommand{\argmin}{\mathop{\arg\min}}
\newcommand{\minimize}{\mathop{\mathbf{minimize}}}

\newcommand{\Vones}[1]{\ensuremath{\mathbf{1}_{#1}}}
\newcommand{\eqdef}{\stackrel{\rm def}{=}}

\newcommand{\out}[1]{}
\newcommand{\denselist}{\itemsep 0pt\topsep-8pt\partopsep-8pt}
\newcommand{\mypar}[1]{\noindent{\bf #1}}

\newcommand{\deriv}[2]{ \frac{\partial #1}{\partial #2} }
\newcommand{\diag}{ \mathbf{diag} }

%% paper-specific definitions:
\newcommand{\w}{\mathbf{w}}
\newcommand{\f}{\mathbf{f}}

\def\naive{na\"{\i}ve\xspace}
\def\Naive{Na\"{\i}ve\xspace}
\def\naively{na\"{\i}vely\xspace}
\def\Naively{Na\"{\i}vely\xspace}

\newcommand{\CPS}{CPS\xspace}
\newcommand{\LLPS}{LLPS\xspace}
\newcommand{\LLPSlong}{Local Linear Pictorial Structures\xspace}


% some common mathcals
\newcommand{\cH}{\mathcal{H}}
\newcommand{\cC}{\mathcal{C}}
\newcommand{\cD}{\mathcal{D}}
\newcommand{\cL}{\mathcal{L}}
\newcommand{\cX}{\mathcal{X}}
\newcommand{\cY}{\mathcal{Y}}
\newcommand{\cR}{\mathcal{R}}
\newcommand{\cE}{\mathcal{E}}
\newcommand{\cV}{\mathcal{V}}

\newcommand{\reals}{\mathbb{R}}
\newcommand{\defn}{\triangleq}

\newcommand{\tree}{\Upsilon}
\newcommand{\attrib}[1]{ \nopagebreak{\raggedleft\footnotesize #1\par}}
\newcommand{\myquotation}[2]{{\em #1}\\\attrib{#2}}

\newcommand{\secref}[1]{\hyperref[sec:#1]{\textsection\ref{sec:#1}}}
\newcommand{\equref}[1]{\hyperref[eq:#1]{Equation~\ref{eq:#1}}}
\newcommand{\algref}[1]{\hyperref[alg:#1]{Algorithm~\ref{alg:#1}}}
\newcommand{\probref}[1]{\hyperref[prob:#1]{Problem~\ref{prob:#1}}}
\newcommand{\thmref}[1]{\hyperref[thm:#1]{Theorem~\ref{thm:#1}}}
\newcommand{\lemref}[1]{\hyperref[lem:#1]{Lemma~\ref{lem:#1}}}
\newcommand{\tabref}[1]{\hyperref[tab:#1]{Table~\ref{tab:#1}}}
\newcommand{\figref}[1]{\hyperref[fig:#1]{Figure~\ref{fig:#1}}}
\newcommand{\figreff}[2]{\hyperref[fig:#1]{Figure~\ref{fig:#1}#2}}


\newcommand{\score}[0]{s(x,y)}
\newcommand{\pscore}[0]{s^p(x,y)}
\newcommand{\mmi}[0]{s^\star_x(y_i)}
\newcommand{\witnessi}[0]{y^\star(y_i)}

%\renewcommand{\includegraphics}[2]{}

%% usual commands
\newcommand{\todo}[1]{\textcolor{red}{{\bf TODO:} #1 }}
%\newcommand{\todo}[1]{{\bf{TODO: #1}}}
%\newcommand{\todo}[1]{}

%\newtheorem{theorem}{Theorem}[section]
%\newtheorem{lemma}[theorem]{Lemma}
%\newtheorem{proposition}[theorem]{Proposition}
%\newtheorem{corollary}[theorem]{Corollary}

%\newenvironment{proof}[1][Proof]{\begin{trivlist}
%\item[\hskip \labelsep {\bfseries #1}]}{\end{trivlist}}
%\newenvironment{definition}[1][Definition]{\begin{trivlist}
%\item[\hskip \labelsep {\bfseries #1}]}{\end{trivlist}}
%\newenvironment{example}[1][Example]{\begin{trivlist}
%\item[\hskip \labelsep {\bfseries #1}]}{\end{trivlist}}
%\newenvironment{remark}[1][Remark]{\begin{trivlist}
%\item[\hskip \labelsep {\bfseries #1}]}{\end{trivlist}}

%\newcommand{\qed}{\nobreak \ifvmode \relax \else
%      \ifdim\lastskip<1.5em \hskip-\lastskip
%      \hskip1.5em plus0em minus0.5em \fi \nobreak
%      \vrule height0.75em width0.5em depth0.25em\fi}

%\newcommand{\qed}{\hfill \ensuremath{\Box}}

%\newcommand{\qed}{\ensuremath{\Box}}

\newcommand{\boxedequation}[2]{%
  \[\fbox{
      \addtolength{\textwidth}{-2\fboxsep}%
      %\addtolength{\linewidth}{-2\fboxsep}%
      \addtolength{\linewidth}{-2\fboxrule}%
      \begin{minipage}{#1\linewidth}%
      \begin{equation}#2\end{equation}%
      \end{minipage}%
    }\]%
}
\newcommand{\boxedeqnarray}[1]{%
  \[\fbox{%
      %\addtolength{\linewidth}{-2\fboxsep}%
      %\addtolength{\linewidth}{-2\fboxrule}%
      \begin{minipage}{0.7\linewidth}%
      \begin{eqnarray*}#1\end{eqnarray*}%
      \end{minipage}\nonumber%
    }\]%
}

%abbreviations
% Add a period to the end of an abbreviation unless there's one
% already, then \xspace.
\makeatletter
\DeclareRobustCommand\onedot{\futurelet\@let@token\@onedot}
\def\@onedot{\ifx\@let@token.\else.\null\fi\xspace}
%\def\eg{\emph{e.g.}}
\def\eg{\emph{e.g}\onedot} \def\Eg{\emph{E.g}\onedot}
\def\ie{\emph{i.e}\onedot} \def\Ie{\emph{I.e}\onedot}
\def\cf{\emph{c.f}\onedot} \def\Cf{\emph{C.f}\onedot}
\def\etc{\emph{etc}\onedot} \def\vs{\emph{vs}\onedot}
\def\wrt{w.r.t\onedot} \def\dof{d.o.f\onedot}
\def\etal{\emph{et al}\onedot}

%\def\ie{\emph{i.e.}}
%\makeatother


\renewcommand{\algorithmicrequire}{\textbf{Input:}} 
\renewcommand{\algorithmicensure}{\textbf{Ouput:}}

%% specific commands
\newcommand{\trans}[1]{{#1}^{\ensuremath{\mathsf{T}}}}           % transpose
\newcommand{\st}{\quad\textrm{s.t.}\quad}                        % s.t. for such that
\newcommand{\mvec}{\textrm{vec}}
\newcommand{\nchoosek}[2]{\left(\begin{array}{c}#1\\#2\end{array}\right)}  %use binom instead...
